%! Author = emil.kero
%! Date = 2023-03-15
\documentclass[11p]{article}
% Packages
\usepackage{amsmath}
\usepackage{graphicx}
\usepackage[swedish]{babel}
\usepackage[
    backend=biber,
    style=authoryear-ibid,
    sorting=ynt
]{biblatex}
\usepackage[utf8]{inputenc}
\usepackage[T1]{fontenc}
%Källor
\addbibresource{mall.bib}
\graphicspath{ {./images/} }

\title{Solenergi \\ \small Fysik 1}
\author{Emil Kero }
\date{\today}

\begin{document}

    \begin{titlepage}
        \begin{center}
            \vspace*{1cm}

            \Huge
            \textbf{Solenergi}

            \vspace{0.5cm}
            \LARGE


            \vspace{1.5cm}

            \textbf{Emil Kero}

            \vfill

            Ett PM om energiförsörjning \\
            Fysik 1

            \vspace{0.8cm}

            \includegraphics[width=0.4\textwidth]{../src/NTI Gymnasiet_Symbol_print_svart.png}

            \Large
            Teknikprogrammet\\
            NTI Gymnasiet\\
            Umeå\\
            \today

        \end{center}
    \end{titlepage}
% Om arbetet är långt har det en innehållsförteckning, annars kan den utelämnas
    \tableofcontents
    \newpage

    \section{Inledning}
    I dagens samhälle är diskussionen kring utvinning av energi mer förekommande än någonsin, och solenergi är något som ofta tas upp som ett exempel på hållbar energi.
    \newline Syftet med detta PM är att redovisa hur och varför solenergi används, och vad för påverkan det har på miljön och samhället.
    \subsection{Frågeställningar}
    Detta PM kommer att sikta på att besvara följande frågor.
    \begin{enumerate}
        \item Hur utvinns solenergi och hur fungerar det?
        \item Vad har användningen av solenergi för påverkan på miljön?
        \item Vad har användningen av solenergi för påverkan på samhället?
    \end{enumerate}

    \section{Resultat}
    \subsection{Så fungerar solenergi}
    Solens strålar tillför varje dag mer än 10 000 gånger mer energi till världen än vad vi faktiskt använder.
    Detta är väldigt mycket energi, men för att faktiskt kunna använda det måste vi fånga upp energin på något vis.
    Det gör vi med solceller. \parencite{Naturskyddsforeningen}
    \newline En solcell fungerar genom att ett tunt lager gjort av ett halvledarmaterial plockar upp solens strålar och omvandlar solenergin till elektricitet.
    Det bildas då en spänning mellan solcellens framsida och baksida, och om man kopplar samman dem får man en elektrisk ström. \parencite{Naturskyddsforeningen}
    \newline Vanligtvis kopplas flera solceller ihop för att bilda en solpanel.
    Dessa placeras oftast på hustak eller på marken i speciellt skapade solparker.
    \subsection{Solenergins påverkan på miljön}
    Även fast det kan verka som att solenergi är den perfekta energikällan så finns det ett stort problem.
    Detta ligger i materialen som används för att skapa solceller.
    Solceller innehåller sällsynta, finita mineraler och metaller vilket kan göra det svårt att massproducera dem. \parencite{Naturskyddsforeningen}
    \newline I sig själv har dock inte solceller egentligen någon påverkan på miljön alls.
    Därför forskas det ständigt om nya sätt att skapa solceller. \parencite{Naturskyddsforeningen}
    \subsection{Solenergins påverkan på samhället}
    En av de stora frågorna kring användning av solceller är hur mycket utrymme de måste ta upp.
    För att utvinna så mycket solenergi som möjligt måste så stor yta av jorden som möjligt täckas, vilket såklart kan vara problematiskt.
    \newline Den mest effektiva lösningen är att placera solcellerna på byggnaders tak.
    Det betyder att ingen markyta behöver användas endast för solceller, och det har bonusen att elektriciteten kan tillföras till byggnaderna direkt från solcellerna. \parencite{Naturskyddsforeningen}

    \section{Sammanfattning}
    Solen är en energikälla som är extremt hållbar och effektiv, men med det stora problemet att energin inte kan utvinnas fullständigt.
    Ifall forskare hittar ett sätt att utvinna solenergi utan att behöva använda massvis av finita material så finns det en stor sannolikhet att solenergi blir den absolut främsta energikällan i framtiden.

    \printbibliography

\end{document}
