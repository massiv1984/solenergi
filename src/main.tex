%! Author = emil.kero
%! Date = 2023-03-15
\documentclass[11p]{article}
% Packages
\usepackage{amsmath}
\usepackage{graphicx}
\usepackage[swedish]{babel}
\usepackage[
    backend=biber,
    style=authoryear-ibid,
    sorting=ynt
]{biblatex}
\usepackage[utf8]{inputenc}
\usepackage[T1]{fontenc}
%Källor
\addbibresource{mall.bib}
\graphicspath{ {./images/} }

\title{Solenergi \\ \small Fysik 1}
\author{Emil Kero }
\date{\today}

\begin{document}

    \begin{titlepage}
        \begin{center}
            \vspace*{1cm}

            \Huge
            \textbf{Solenergi}

            \vspace{0.5cm}
            \LARGE


            \vspace{1.5cm}

            \textbf{Emil Kero}

            \vfill

            Ett PM om energiförsörjning \\
            Fysik 1

            \vspace{0.8cm}

            \includegraphics[width=0.4\textwidth]{../src/NTI Gymnasiet_Symbol_print_svart.png}

            \Large
            Teknikprogrammet\\
            NTI Gymnasiet\\
            Umeå\\
            \today

        \end{center}
    \end{titlepage}
% Om arbetet är långt har det en innehållsförteckning, annars kan den utelämnas
    \tableofcontents
    \newpage

    \section{Inledning}
    I dagens samhälle är diskussionen kring utvinning av energi mer förekommande än någonsin, och solenergi är något som ofta tas upp som ett exempel på hållbar energi.
    \newline Syftet med detta PM är att redovisa hur och varför solenergi används, och vad för påverkan det har på miljön och samhället.
    \subsection{Frågeställningar}
    Detta PM kommer att sikta på att besvara följande frågor.
    \begin{enumerate}
        \item Hur utvinns solenergi och hur fungerar det?
        \item Vad har användningen av solenergi för påverkan på miljön?
        \item Vad har användningen av solenergi för påverkan på samhället?
    \end{enumerate}

    \section{Resultat}
    \subsection{Så fungerar solenergi}
    \subsection{Solenergins påverkan på miljön}
    \subsection{Solenergins påverkan på samhället}

    \section{Slutsatser}
    Här kan du dra slutsatser eller sammanfatta ditt resultat

% Mer saker som du kan ha nytta av.

    \section{Referenser}
    Referenser i text kan skrivas på två sätt: Enligt \textcite{Jens} kan man använde två typer av referenser, inbäddade i texten eller efter ett fakta \parencite{Fraenkel}. Ett till test för att se hur det ser ut \parencite[sid 55]{fermi}.

    \section{Annat som kan vara bra att veta}
    Om du vill ha kodstil och få med alla tecken kan du använda verbatim. då kan du skriva \verb|abcd!"#| utan problem...

    Citat skrivs mellan de konstiga symbolerna \verb|``| och \verb|''| för att de ska se bra ut ``se bra ut!''.
    \subsection{En underrubrik}
    \subsubsection{En underunderrubrik}
    \subsection{Ekvationer}
    Det är lätt att skriva matematik i \LaTeX

    \begin{equation}
        F = G \frac{M m}{r^2}
        \label{grav}
    \end{equation}

    Ekvation (\ref{grav}) känner ni igen...

    \subsection{figurer}
    Bilder placeras enklast på detta sätt. placeringen bestämmer \LaTeX och vi kan bara föreslå (h)är, (t)opp eller (b)otten. Ett utropstecken före tvingar lite mer men inte absolut. I bild \ref{varg} visas en varg
    \begin{figure}[!h]
        \includegraphics[width=0.8\textwidth]{../images/accelerationTime.png}
        \caption{Acceleration-tid diagram. Källa: Impuls Fysik 1}
        \label{varg}
    \end{figure}
    \printbibliography

\end{document}
